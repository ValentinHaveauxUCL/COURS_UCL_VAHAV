\PassOptionsToPackage{unicode=true}{hyperref} % options for packages loaded elsewhere
\PassOptionsToPackage{hyphens}{url}
%
\documentclass[]{article}
\usepackage{lmodern}
\usepackage{amssymb,amsmath}
\usepackage{ifxetex,ifluatex}
\usepackage{fixltx2e} % provides \textsubscript
\ifnum 0\ifxetex 1\fi\ifluatex 1\fi=0 % if pdftex
  \usepackage[T1]{fontenc}
  \usepackage[utf8]{inputenc}
  \usepackage{textcomp} % provides euro and other symbols
\else % if luatex or xelatex
  \usepackage{unicode-math}
  \defaultfontfeatures{Ligatures=TeX,Scale=MatchLowercase}
\fi
% use upquote if available, for straight quotes in verbatim environments
\IfFileExists{upquote.sty}{\usepackage{upquote}}{}
% use microtype if available
\IfFileExists{microtype.sty}{%
\usepackage[]{microtype}
\UseMicrotypeSet[protrusion]{basicmath} % disable protrusion for tt fonts
}{}
\IfFileExists{parskip.sty}{%
\usepackage{parskip}
}{% else
\setlength{\parindent}{0pt}
\setlength{\parskip}{6pt plus 2pt minus 1pt}
}
\usepackage{hyperref}
\hypersetup{
            pdfborder={0 0 0},
            breaklinks=true}
\urlstyle{same}  % don't use monospace font for urls
\setlength{\emergencystretch}{3em}  % prevent overfull lines
\providecommand{\tightlist}{%
  \setlength{\itemsep}{0pt}\setlength{\parskip}{0pt}}
\setcounter{secnumdepth}{0}
% Redefines (sub)paragraphs to behave more like sections
\ifx\paragraph\undefined\else
\let\oldparagraph\paragraph
\renewcommand{\paragraph}[1]{\oldparagraph{#1}\mbox{}}
\fi
\ifx\subparagraph\undefined\else
\let\oldsubparagraph\subparagraph
\renewcommand{\subparagraph}[1]{\oldsubparagraph{#1}\mbox{}}
\fi

% set default figure placement to htbp
\makeatletter
\def\fps@figure{htbp}
\makeatother


\date{}

\begin{document}

\hypertarget{note-du-cours-de-concept-de-language-informatique}{%
\section{Note du cours de ``Concept de language
informatique''}\label{note-du-cours-de-concept-de-language-informatique}}

\emph{07 Mars 2018} \emph{LINGI 1131} \emph{Valentin Haveaux}

\hypertarget{note-tierce}{%
\subsubsection{Note tierce}\label{note-tierce}}

\begin{quote}
Le prof veut faire une intero comme il l'avait fait pour le cours de
logique. Il a demandé que les délégués lui envoie la date qui
conviendrait le mieux (sans doute dans 2 semaines).
\end{quote}

\hypertarget{contenu-du-cours}{%
\subsection{Contenu du cours}\label{contenu-du-cours}}

\begin{itemize}
\tightlist
\item
  Lazy suspensions

  \begin{itemize}
  \tightlist
  \item
    Hamming problem
  \item
    Lazy Quicksort
  \end{itemize}
\item
  Declarative concurrency

  \begin{itemize}
  \tightlist
  \item
    What is declarative concurrency?
  \item
    Limitations of declarative conc.
  \end{itemize}
\item
  Message-passing concurrency

  \begin{itemize}
  \tightlist
  \item
    Ports
  \end{itemize}
\end{itemize}

\hypertarget{lazy-suspensions}{%
\subsection{Lazy suspensions}\label{lazy-suspensions}}

\begin{verbatim}
Fun lazy {F X}                          Z = {F X}
    <e>                                 --> Create a thread
end
    |
    |                                   
    V
proc {F X R}                            Z lazy suspension
    thread                                  {F X}
        {WaithNeeded R}
        R = <e>
    end
end
\end{verbatim}

\hypertarget{hamming-problem}{%
\subsubsection{Hamming problem}\label{hamming-problem}}

\begin{itemize}
\item
  Generate stream of integers

\begin{verbatim}
  a b c
  2 3 5
  a,b,c >= 0
\end{verbatim}

  in invesing order
\item
  Start with 1 1 2 3 4 5 6 8 9 10 12 \ldots{}
\item
  Lenght of stream \textbf{not know in advence}
\end{itemize}

idea: generate infinite stream \textbf{lazily}

\begin{verbatim}
H = 1|.........|hi|......|hn|.......
                           |    |
                           |    V
                           V    hn + 1 ?
                        last to fan


             ______________________________________
            |          ____________________________|
            |         |          __________________|
            | 5*h2    | 3*h2    | 2*h2             |
            V         V         V                  |
#############################################  #########
# 1 #######   #######   #######   #######   #  ###   ###
#############################################  #########
ho                                         hn    hn+1
\end{verbatim}

\begin{verbatim}
%Hamming problem
declare
H=1|{Merge {Times H 2}}
    {Merge {Times H 3} {Times H 5}}}
    
{Browse H}
{Browse H.2.1}

declare
proc {Touch H N}
    if N==0 then skip
    else {Touch H.2 N-1}
    end
end

{Touch H 5}
{Touch H 10}
{Touch H 100}
{Touch H 500}

declare
fun lazy {Times H N}
    case H of E|H2 then E*N|{Times H2 N} end
end

declare
fun lazy {Merge S1 S2}
    case S1|S2
    of (E1|T1)|(E2|T2) then
        if E1<E2 then
            E1|{Merge T1 S2}
        elseif E1>E2 then
            E2|{Merge S1 T2}
        else /* E1==E2 */
            E1|{Merge T1 T2}
        end
    end
end
\end{verbatim}

\begin{itemize}
\tightlist
\item
  Start execution
\end{itemize}

\begin{quote}
H = 1 \textbar{} H' ---\textgreater{} \{Times H 2\} ---\textgreater{}
Lazy suspension \textbar{} \textbar{} ---\textgreater{} \{Merge . .\}
\textbar{} ---\textgreater{} \{Times H 3\} \textbar{} \textbar{}
---\textgreater{} \{Merge . .\} \textbar{} \textbar{} ---\textgreater{}
\{Times H 5\}
\end{quote}

\begin{quote}
\hypertarget{h}{%
\subsection{H'}\label{h}}

\{Browse H.2.1\} (second element) -\textgreater{} H' needed \textbar{} a
activating\\
-\textgreater{} execute \{Merge S1 S2\} \textbar{} the lazy expression
-\textgreater{} body: case S1\textbar{}S2 of
(E1\textbar{}T1)\textbar{}(E2\textbar{}T2) then \ldots{}. \textbar{}
\textbar{}\\
end V \textbar{} need S1 V need S2
\end{quote}

\hypertarget{lazy-quicksort}{%
\subsubsection{Lazy Quicksort}\label{lazy-quicksort}}

\begin{itemize}
\item
  Quicksort

\begin{verbatim}
          ########################
          # X #                  #  L
          ########################
              |           |
              |           |
              V           V
      #############   #############
   L1 #           # L2#           #
      #############   #############
      all elem < X     all elem >= X
          |                 |
          V  Q              V  Q
      #############   #############
   S1 #           # S2#           #
      #############   #############
        sorted            sorted
              |           |
              |           |
              V           V
          #######################
       S  # # # # # # # # # # # #  
          #######################
\end{verbatim}
\end{itemize}

\begin{verbatim}
%Quicksort

%Partition
declare
proc {Partition L X L1 L2}
    case L of Y|M then
        if Y<X then M1 in
            L1=Y|M1 {Partition M X M1 L2}
        else M2 in
            L2=Y|M2 {Partition M X L1 M2}
        end
    [] nil then L1=nil L2=nil
    end
end

declare L L1 L2 in
L=[2 7 5 6 4 5 4 3 5]
{Partition L 4 L1 L2}
{Browse L1}
{Browse L2}

% First version
/*
declare 
fun {Append L1 L2}
    case L1 of X|M1 then X|{Append M1 L2}
    [] nil then L2 end
end

declare
fun {Quicksort L}
    case L of X|M then L1 L2 S1 S2 in
        {Partition M X L1 L2}
        S1={Quicksort L1}
        S2={Quicksort L2}
        {Append S1 X|S2}
    [] nil then nil
    end
end

{Browse {Quicksort [6 5 3 78 45 43 2 1 65 4 7]}}
*/


% Lazy version

declare 
fun lazy {LAppend L1 L2}
    case L1 of X|M1 then X|{LAppend M1 L2}
    [] nil then L2 end
end

declare
fun lazy {LQuicksort L}
    case L of X|M then L1 L2 S1 S2 in
        {Partition M X L1 L2}
        S1={LQuicksort L1}
        S2={LQuicksort L2}
        {LAppend S1 X|S2}
    [] nil then nil
    end
end

declare L S in
L=[3 6 2 5 1 7 0]
S={Quicksort L}
{Browse S}

{Browse S.1}
{Browse S.2.1}
{Touch S 7}
{Touch S 8}

% Execution when I ask for least element
declare L S in
L=[]
\end{verbatim}

\end{document}
